\section{Case Study: Coupled Path Traced Machine Learning}

Utilisation of PAVE consists of three consecutive phases: rendering phase of conditional training images, training phase of the generative neural network, and execution phase of the trained network. Three core components, VTK-m, PyTorch, and PAVE, each fulfill a unique functional requirement during the separate stages. In this section we describe the independent design and global role each system plays.

\subsection{System Overview}
 
To achieve our goal of a conditional generative neural network capable of rendering geometric dependent object path simulations we begin by rendering informative conditional image buffers along with ground truth scene renderings. For this purpose the VTK-m was chosen due to its scalability and robust capability for HPC visualisation tasks. Provided the training set of conditional and ground truth images two neural networks, one convolutional and one generative, play a zero-sum game common to training GANs. To segue data management of training images the path tracer saves the training set in a distributed setting with the use of PAVE. During training PyTorch is then able to retrieve needed image data through the use of the PAVE.

\subsection{Path Tracer Design}

High quality ground truth rendered scenes and conditional image attributes are required for the training stage. For this reason, the first stage of PAVE consists of generating a visual scene or simulation with VTK-m. 

Within the parallel primitives framework of VTK-m, a path tracer was implemented to renders scenes. A path tracer utilizes Monte Carlo sampling through probability distribution functions over shapes of interest and light intensity and pixel values with cumulative distribution function sampling, light scattering, randomly directed light paths and material sampling to generate realistic lighting for a scene. 

Further, the image buffers needed to compute light paths afford an informative conditional dependency on the behavior of lighting based on the geometry and light sources within a scene. The VTK-m ray tracer~\cite{95260c8902184519bd98df42d2572515} was adapted to render these conditional buffers, namely albedo, direct lighting, normals of surfaces and depth with respect to point of view are then stored or passed to PyTorch with PAVE, allowing for a a pipeline which preserves the solutions scalability. 

\subsection{Neural Network Design}

The cGAN used closely follows that introduced by Thomas and Forbes with Deep Illumination \cite{deepillum}. Both the discriminator and generator network are deep convolutional neural networks implemented in PyTorch using training data retrieved from Adios files formatted and stored by the VTK-m path tracer. The training stage relies on four conditional buffers depth, albedo, normals and direct lighting along with an associated ground truth image of high light sample count and ray depth. Given the four conditional buffers the generator attempts to construct the ground truth image from noise. The discriminator is then fed both the generated and ground truth image. The loss used for the gradient back propagation update of both networks is based on the quality of the discriminators ability to classify the artificial and true image in which the generator is greater penalised when the discriminator accurately differentiates the two images, and similarly, the discriminator has a larger loss when incorrectly identifying real from fabricated images. The generator is then considered to have converged when the discriminator predicts both generated and true images with equal probability. For both discriminator and generator networks the activation functions used between layers is LeakyReLu and Sigmoid for the final layer \cite{maasLeaky}. Batch normalisation is also performed between internal layers to minimise covariant shift of weight updates and improve learning for the deeper networks used \cite{ioffeBatch}.

\subsubsection{Generator Network}

The generative network used is a deep convolutional network consisting of an encoder and decoder with skip connections-concatenations of equal depth layers within the encoding and decoding stages. Due to the illustrative `shape' of this design the network is denoted a U-Net as introduced by Ronneberger et. al. for medical segmentation \cite{ronnebergerUnet}. The motivation for utilising a U-Net is due to success of the skip connections in capturing geometric and spatial attributes by linking the decoded convolutional process to the encoded upconvolutional. The mapping of contracting feature segmentation onto expanding upsampling within the network allows us to also exploit nearness to object geometry within the constructed and target image through Euclidean distance. As a result, performance in terms of required training time, quality of preserved structure, and accuracy maintaining light information of generated images drastically improves with the addition of an L1 loss to the classic binary cross entropy common to training adversarial networks when updating the discriminator and generator  \cite{isolaL1}\cite{goodfellowGAN}. 

\subsubsection{Discriminator Network}

For discriminating between artificial and ground truth image renderings a deep convolutional patchGAN network is used motivated by the added advantage of providing a patch-wise probability of an image in question as being real or fake. The benefit of a patch-wise probability allows for higher regional accuracy within an image as well as applicable for image-to-image tasks as introduced by Isola et. al. \cite{isolaPatch}. The image classification probability is then interpreted as the average of these patch wise probabilities over an image in question. 

As input during training the discriminator network is given the set of conditional space buffers along with either the visualisation generated by the cGAN generator network or the ground truth global illumination rendering produced with the VTK-m path tracer. Taking into account the conditional image buffers stacked atop an image sample an image sample under question the resulting input is a tensor of the form Width x Height x 15. The discriminator then attempts to predict with what probability the provided image stack, either fabricated by the U-Net or the VTK-m path tracer, is real. Based on the discriminators performance the loss is computed using the classic loss for training GANs along with an L1 loss. Training is complete, and the generator has converged, when the discriminator predicts images as real or fake with an equal probability, e.g. 50\% chance of accuracy. At this point the discriminator network is discarded and the resulting generator affords a real-time in situ visualisation tool able to produce accurate global illumination from conditional geometric buffers.
